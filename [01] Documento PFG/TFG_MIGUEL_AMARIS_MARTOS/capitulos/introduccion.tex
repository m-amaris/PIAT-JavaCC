\section{Contexto y motivación}
El presente Proyecto de Fin de Grado surge del interés personal por el procesamiento de información textual y la búsqueda de herramientas versátiles y eficientes para su análisis. Desde una temprana edad, me ha fascinado la capacidad del lenguaje para construir mundos, comunicar ideas y conectar con otras personas. En este sentido, la informática me ha brindado la oportunidad de explorar el lenguaje desde una perspectiva más técnica y creativa, permitiéndome comprender cómo las máquinas pueden procesar y generar información textual.

A lo largo de mi formación académica, he tenido la oportunidad de aprender sobre diferentes herramientas para el procesamiento de información, como las clases de Java para JSON o XML. Sin embargo, estas herramientas se limitan a un formato específico, lo que implica la necesidad de aprender un nuevo lenguaje o API para cada caso. Esta limitación me motivó a buscar una alternativa más universal que me permitiera abordar cualquier tipo de formato textual sin necesidad de un aprendizaje adicional.

En este contexto, descubrí JavaCC, un generador de analizadores léxicos y sintácticos que me fascinó por su potencia y flexibilidad\cite{javaccgithub}. Esta herramienta me abrió un mundo de posibilidades al permitirme concebir una forma de procesar información de forma genérica, independientemente del formato en el que se presente.

Este proyecto se presenta como una oportunidad excepcional para profundizar en mi conocimiento del lenguaje y explorar sus aplicaciones en el ámbito del procesamiento de información. Además, me motiva la idea de contribuir a la asignatura de PIAT con una documentación completa y accesible que facilite el aprendizaje de esta herramienta a otros estudiantes.

Más allá de las ventajas técnicas, este proyecto me permite conectar con mi pasión por el lenguaje y explorar su potencial en el mundo digital. Aspiro a convertirme en un experto en el procesamiento de información textual, y este proyecto es un paso fundamental en ese camino.

Estoy convencido de que este proyecto le aportará al lector conocimientos técnicos valiosos, además de habilitarle a desarrollar sus capacidades de aprendizaje autónomo, investigación y creatividad. 

En definitiva, este proyecto representa una oportunidad única para combinar mi pasión por el lenguaje con mi formación en informática, permitiéndome contribuir al desarrollo de nuevas herramientas y conocimientos en el ámbito del procesamiento de información textual.

\section{Objetivos}

El objetivo principal del proyecto es evaluar la viabilidad de utilizar una herramienta de generadores genéricos como JavaCC para abordar las prácticas de PIAT. 
Esto implica: 
\begin{itemize}
    \item Aprender a utilizar JavaCC de manera efectiva. 
    \item Aplicar JavaCC en prácticas específicas de PIAT. (Quitar los bucles y los loops de piat) 
    \item Generar documentación que sirva como recurso para estudiantes y profesores interesados en utilizar JavaCC en proyectos relacionados con el procesamiento de información. 
\end{itemize}

El proyecto busca proporcionar una solución versátil y eficaz para el análisis de gramáticas y el procesamiento de información en el ámbito de PIAT

\section{Restricciones}
	En la realización del proyecto, debemos tener en cuenta las siguientes restricciones:
\begin{itemize}
    \item El proyecto se centrará en la utilización de JavaCC como herramienta principal. 
    \item Debe garantizarse que la documentación generada sea comprensible y útil para aquellos que deseen aplicar JavaCC en proyectos similares.
    \item Las prácticas de PIAT deben servir como un contexto relevante para la aplicación de JavaCC.
\end{itemize}

\section{Estructura del documento}