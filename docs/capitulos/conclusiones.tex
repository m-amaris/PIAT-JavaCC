\noindent Este Trabajo de Fin de Grado se propuso evaluar el potencial de JavaCC como herramienta única para el procesamiento de información en el contexto de la asignatura PIAT, explorando su viabilidad como alternativa a las herramientas tradicionalmente utilizadas, como SAX, GSON y XPath. Para llevar a cabo esta evaluación, se implementaron cuatro prácticas de PIAT utilizando JavaCC, abordando el análisis de archivos de log, el procesamiento de documentos XML y la gestión de datos en formato JSON.

La segunda práctica, centrada en el análisis de archivos de log de un sistema de correo electrónico, demostró la eficiencia de JavaCC en la extracción de información estadística. Se implementó un analizador léxico y sintáctico que, en comparación con el uso de expresiones regulares tradicionales, logró una reducción del tiempo de ejecución del 82\%. Esta mejora se atribuye a la eficiencia en la compilación de gramáticas, la menor sobrecarga de procesamiento y la optimización del código generado por JavaCC. La capacidad de definir gramáticas estructuradas y ejecutar acciones específicas según el tipo de traza identificado, convirtieron a JavaCC en una herramienta mucho más eficiente para el análisis de grandes volúmenes de datos.

En la tercera práctica, se abordó el procesamiento de archivos XML, implementando un analizador basado en JavaCC para extraer información específica de un conjunto de datos del Ayuntamiento de Madrid. La principal ventaja observada fue la simplicidad en la generación del analizador y su capacidad para manejar recursividad en los documentos XML de forma eficiente. A diferencia de SAX, que requeriría una modificación sustancial del código ante un cambio en la profundidad de la recursividad del archivo XML, JavaCC permite definir la gramática de manera que se adapte a diferentes niveles de anidamiento sin necesidad de reescribir el código principal, aportando flexibilidad y facilitando el mantenimiento del código.

% \textbf{[Aquí incluir la información sobre la práctica 4, describiendo los resultados obtenidos y cómo JavaCC se compara con GSON en este caso particular. Destacar las ventajas y desventajas de usar JavaCC en este contexto].}

La cuarta práctica, centrada en el análisis de archivos JSON, demostró la capacidad de JavaCC para realizar extracciones selectivas de información, optimizando el proceso y mejorando el rendimiento en comparación con GSON Streaming. Se implementó un analizador que, al definir estados léxicos específicos, permitió ignorar elementos irrelevantes del JSON y centrarse en la información deseada. Esta optimización se tradujo en una reducción significativa del tiempo de procesamiento, especialmente al tratar con archivos JSON de gran tamaño. La principal ventaja de JavaCC en este contexto es su control granular sobre el proceso de análisis, permitiendo al desarrollador definir qué elementos son relevantes y cómo se deben procesar. Sin embargo, la definición de gramáticas y estados léxicos puede ser más compleja que la utilización de GSON Streaming, que ofrece una API más sencilla para la lectura secuencial de JSON.


% \textbf{[Aquí incluir la información sobre la práctica 5, describiendo los resultados obtenidos y cómo JavaCC se compara con XPath en este caso particular. Resaltar las ventajas y desventajas de usar JavaCC en este contexto].}

La quinta práctica, centrada en la transformación de documentos XML a JSON, consolidó la viabilidad de JavaCC como herramienta única para el análisis y procesamiento de información en PIAT. Se implementó un analizador que, al integrar el análisis léxico, el análisis sintáctico, la extracción de información y la generación de documentos en un único framework, simplificó el desarrollo y ofreció un mayor control sobre el proceso. Si bien en esta práctica la diferencia en rendimiento entre JavaCC y XPath no fue tan significativa como en la práctica 2, la simplicidad de la gramática JavaCC y su capacidad para adaptarse a cambios en la estructura del archivo XML la convierten en una alternativa atractiva a largo plazo. La principal ventaja de JavaCC en este contexto es su flexibilidad y control sobre el proceso de análisis, permitiendo al desarrollador definir con precisión qué información se debe extraer y cómo se debe estructurar el JSON resultante. Sin embargo, la curva de aprendizaje de JavaCC puede ser más pronunciada que la de XPath, especialmente para usuarios no familiarizados con la teoría de los analizadores sintácticos.

En resumen, las cuatro prácticas han permitido demostrar la versatilidad y eficiencia de JavaCC como herramienta única para el procesamiento de información en PIAT. Si bien existen herramientas especializadas que pueden resultar más eficientes en casos específicos, la versatilidad, eficiencia y capacidad de JavaCC para manejar estructuras complejas lo convierten en una alternativa a considerar, especialmente cuando se requiere un alto grado de control sobre el proceso de análisis o se necesita una solución adaptable a diferentes formatos y estructuras de datos. La capacidad de definir gramáticas, la eficiencia en la compilación y la optimización del código generado, convierten a JavaCC en una alternativa sólida y eficiente a las herramientas tradicionales utilizadas en la asignatura.


% \textbf{\#TODO: completar la sección comentando que en lugar de centrarse en tecnologías concretas, trabajar con tecnologías diversas, con javacc tu tienes un enfoque genérico independiente de la tecnología... }



La utilización de JavaCC en las prácticas de PIAT ha demostrado la viabilidad de un enfoque genérico para el procesamiento de información, independiente de la tecnología o formato específico de los datos. En lugar de depender de herramientas especializadas para cada formato, como SAX para XML o GSON para JSON, JavaCC permite crear analizadores personalizados que se adaptan a las necesidades específicas de cada caso. Este enfoque genérico no solo simplifica el proceso de desarrollo, sino que también fomenta la adaptabilidad y el pensamiento crítico en los estudiantes, preparándolos para afrontar los desafíos del análisis y procesamiento de información en un mundo cada vez más diverso y complejo. Al abstraer la lógica del análisis sintáctico y léxico en una gramática independiente del formato, JavaCC se convierte en una herramienta versátil y potente para abordar una amplia gama de problemas de procesamiento de información, liberando al desarrollador de las limitaciones impuestas por las herramientas especializadas y permitiéndole enfocarse en la lógica de la aplicación.

\phantom{text}

\noindent \textbf{Líneas futuras de trabajo:}

\phantom{text}

Este Trabajo de Fin de Grado abre un camino para futuras investigaciones y desarrollos en torno a la aplicación de JavaCC en el procesamiento de información. Algunas líneas de trabajo futuro podrían ser:

Evaluación comparativa más exhaustiva: Realizar un estudio comparativo más profundo entre JavaCC y otras herramientas de análisis de datos, considerando un mayor número de casos de uso y métricas de evaluación.
Desarrollo de herramientas auxiliares: Explorar la creación de herramientas y bibliotecas que faciliten la integración de JavaCC en el desarrollo de aplicaciones web y móviles.
Aplicación en otros contextos educativos: Investigar la viabilidad de utilizar JavaCC en otras asignaturas o proyectos dentro del ámbito de la ingeniería informática y las telecomunicaciones.
En definitiva, este proyecto ha sentado las bases para un uso más amplio y eficiente de JavaCC en el procesamiento de información, abriendo un abanico de posibilidades para futuras investigaciones y aplicaciones en el ámbito educativo y tecnológico.