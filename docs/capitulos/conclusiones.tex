\noindent Este Trabajo de Fin de Grado se propuso evaluar el potencial de JavaCC como herramienta única para el procesamiento de información en el contexto de la asignatura PIAT, explorando su viabilidad como alternativa a las herramientas tradicionalmente utilizadas, como SAX, GSON y XPath. Para llevar a cabo esta evaluación, se implementaron cuatro prácticas de PIAT utilizando JavaCC, abordando el análisis de archivos de log, el procesamiento de documentos XML y la gestión de datos en formato JSON.

La segunda práctica, centrada en el análisis de archivos de log de un sistema de correo electrónico, demostró la eficiencia de JavaCC en la extracción de información estadística. Se implementó un analizador léxico y sintáctico que, en comparación con el uso de expresiones regulares tradicionales, logró una reducción del tiempo de ejecución del 450\%. Esta mejora se atribuye a la eficiencia en la compilación de gramáticas, la menor sobrecarga de procesamiento y la optimización del código generado por JavaCC. La capacidad de definir gramáticas estructuradas y ejecutar acciones específicas según el tipo de traza identificado, convirtieron a JavaCC en una herramienta mucho más eficiente para el análisis de grandes volúmenes de datos.

En la tercera práctica, se abordó el procesamiento de archivos XML, implementando un analizador basado en JavaCC para extraer información específica de un conjunto de datos del Ayuntamiento de Madrid. La principal ventaja observada fue la simplicidad en la generación del analizador y su capacidad para manejar recursividad en los documentos XML de forma eficiente. A diferencia de SAX, que requeriría una modificación sustancial del código ante un cambio en la profundidad de la recursividad del archivo XML, JavaCC permite definir la gramática de manera que se adapte a diferentes niveles de anidamiento sin necesidad de reescribir el código principal, aportando flexibilidad y facilitando el mantenimiento del código.

\textbf{[Aquí incluir la información sobre la práctica 4, describiendo los resultados obtenidos y cómo JavaCC se compara con GSON en este caso particular. Destacar las ventajas y desventajas de usar JavaCC en este contexto].}

\textbf{[Aquí incluir la información sobre la práctica 5, describiendo los resultados obtenidos y cómo JavaCC se compara con XPath en este caso particular. Resaltar las ventajas y desventajas de usar JavaCC en este contexto].}

En resumen, las cuatro prácticas han permitido demostrar la versatilidad y eficiencia de JavaCC como herramienta única para el procesamiento de información en PIAT. Si bien existen herramientas especializadas que pueden resultar más eficientes en casos específicos, la versatilidad, eficiencia y capacidad de JavaCC para manejar estructuras complejas lo convierten en una alternativa a considerar, especialmente cuando se requiere un alto grado de control sobre el proceso de análisis o se necesita una solución adaptable a diferentes formatos y estructuras de datos. La capacidad de definir gramáticas, la eficiencia en la compilación y la optimización del código generado, convierten a JavaCC en una alternativa sólida y eficiente a las herramientas tradicionales utilizadas en la asignatura.

\phantom{text}

\noindent \textbf{Líneas futuras de trabajo:}

\phantom{text}

Este Trabajo de Fin de Grado abre un camino para futuras investigaciones y desarrollos en torno a la aplicación de JavaCC en el procesamiento de información. Algunas líneas de trabajo futuro podrían ser:

Evaluación comparativa más exhaustiva: Realizar un estudio comparativo más profundo entre JavaCC y otras herramientas de análisis de datos, considerando un mayor número de casos de uso y métricas de evaluación.
Desarrollo de herramientas auxiliares: Explorar la creación de herramientas y bibliotecas que faciliten la integración de JavaCC en el desarrollo de aplicaciones web y móviles.
Aplicación en otros contextos educativos: Investigar la viabilidad de utilizar JavaCC en otras asignaturas o proyectos dentro del ámbito de la ingeniería informática y las telecomunicaciones.
En definitiva, este proyecto ha sentado las bases para un uso más amplio y eficiente de JavaCC en el procesamiento de información, abriendo un abanico de posibilidades para futuras investigaciones y aplicaciones en el ámbito educativo y tecnológico.