
\noindent En este apéndice se presenta el código fuente de los archivos referenciados a lo largo del documento para su estudio. Cabe destacar que se proporciona dicho código con el objetivo de que el lector sea capaz de probar las funcionalidades y aprender mas acerca de las aplicaciones de JavaCC.

No se promueve el plagio y la realización de las prácticas, este simplemente es una de las muchas formas en las que se podrían realizar la prácticas.

La realización de las prácticas y de todos los ejemplos mostrados en este documento se encuentran disponibles en:

\href{https://github.com/m-amaris/tfg/tree/main/eclipse-workspace}{https://github.com/m-amaris/tfg/tree/main/eclipse-workspace}

% A partir de este punto, no quieres mostrar los números de página
\pagestyle{empty}

% Definimos los nuevos márgenes de esta sección
\newgeometry{left=1cm,right=1cm,top=2cm,bottom=1cm}

\section{Mathexp.jj}
\label{sec:mathexp}
\lstset{inputencoding=utf8/latin1}
\lstinputlisting[language=java]{code/mathexp.jj}

\newpage
\section{NL\_Xlator.jj}
\label{sec:nlxlator}
\lstset{inputencoding=utf8/latin1}
\lstinputlisting[language=java]{code/NL_Xlator.jj}

\newpage
\section{Descripción general del sistema de correo electrónico}
\label{sec:P2SistemaCorreo}

\noindent Esta sección se ha extraído del enunciado de la Práctica 2 de PIAT.

\begin{figure}[H]
	\centering
	\includegraphics[width=\textwidth]{imagenes/P2SistemaCorreo.jpg}
	\caption{\label{fig:P2SistemaCorreo.jpg}Arquitectura del sistema de correo electrónico\cite{pdfpractica2} }
\end{figure}

\noindent Los mensajes entrantes desde Internet son recibidos inicialmente por los servidores smtp-in, los cuales se los pasan a los servidores security-in que actúan de filtros de antispam y virus entrantes. Si estos filtros detectan un virus, descartan el mensaje y no lo reenvían. En otro caso, se los pasan a los servidores de almacenamiento de correos user-mailbox. Los mensajes que se consideren spam se etiquetan con una marca que sirve a los servidores user-mailbox para almacenarlos en la carpeta de posible spam de la cuenta del usuario.

Los servidores security-in generan mensajes automáticos de notificación de error para correos que no pueden llegar al destino. Esto sucede cuando intentan entregar un mensaje a un user-mailbox y éste responde que la dirección de destino es inexistente o bien que la cuenta de destino ya ha alcanzado su cuota máxima de espacio de almacenamiento de mensajes. En ese momento envía un mensaje a smtp-in y se notifica este error al remitente través de los servidores smtp-out.

Los usuarios del servicio de correo electrónico envían nuevos mensajes, y respuestas a los mensajes recibidos, por medio de los servidores msa (mail submission agent). Estos pasan los mensajes a los servidores security-out que actúan de filtros de spam y virus salientes. En el caso de que estos filtros detecten algún mensaje de spam o que contenga un virus, lo bloquean con objeto de salvaguardar la reputación del sistema. Finalmente, los mensajes son entregados a los smtp-out para que estos los envíen a la estafeta de destino.

\newpage
\section{Formato de logs. Sistema de correo electrónico (Simplificado)}
\label{sec:logscorreo}
\lstset{inputencoding=utf8/latin1}
\lstinputlisting[language=java]{code/msa1.2020022016.log}

\newpage
\section{Parser.jj}
\label{sec:P2Parser}
\lstset{inputencoding=utf8/latin1}
\lstinputlisting[language=java]{code/Parser.jj}

\newpage
\section{Catalogo.xml (Simplificado)}
\label{sec:catalogoxml}
\lstset{inputencoding=utf8/latin1}
\lstinputlisting[language=XML]{code/catalogo.xml}

\newpage
\section{Catalogo.xsd }
\label{sec:catalogoxsd}
\lstset{inputencoding=utf8/latin1}
\lstinputlisting[language=XML]{code/catalogo.xsd}

\newpage
\section{XMLParser.jj}
\label{sec:XMLParser}
\lstset{inputencoding=utf8/latin1}
\lstinputlisting[language=java]{code/XMLParser.jj}

\newpage
\section{JSONParser.jj}
\label{sec:JSONParser}
\lstset{inputencoding=utf8/latin1}
\lstinputlisting[language=java]{code/JSONParser.jj}

% Volvemos a la geometría por defecto del documento
\restoregeometry