
En este apéndice se presenta el código fuente de los archivos referenciados a lo largo del documento para su estudio. Cabe destacar que se proporciona dicho código con el objetivo de que el lector sea capaz de probar las funcionalidades y aprender mas acerca de las aplicaciones de JavaCC.

No se promueve el plagio y la realización de las prácticas, este simplemente es una de las muchas formas en las que se podrían realizar la prácticas.

La realización de las prácticas y de todos los ejemplos mostrados en este documento se encuentran disponibles en:

\href{https://github.com/m-amaris/tfg/tree/main/eclipse-workspace}{https://github.com/m-amaris/tfg/tree/main/eclipse-workspace}

% A partir de este punto, no quieres mostrar los números de página
\pagestyle{empty}

% Definimos los nuevos márgenes de esta sección
\newgeometry{left=1cm,right=1cm,top=2cm,bottom=1cm}

\section{Mathexp.jj}
\label{sec:mathexp}
\lstset{inputencoding=utf8/latin1}
\lstinputlisting[language=java]{code/mathexp.jj}

\newpage
\section{NL\_Xlator.jj}
\label{sec:nlxlator}
\lstset{inputencoding=utf8/latin1}
\lstinputlisting[language=java]{code/NL_Xlator.jj}

\newpage
\section{Catalogo.xml (Simplificado)}
\label{sec:catalogoxml}
\lstset{inputencoding=utf8/latin1}
\lstinputlisting[language=XML]{code/catalogo.xml}

\newpage
\section{Catalogo.xsd }
\label{sec:catalogoxsd}
\lstset{inputencoding=utf8/latin1}
\lstinputlisting[language=XML]{code/catalogo.xsd}

\newpage
\section{XMLParser.jj}
\label{sec:XMLParser}
\lstset{inputencoding=utf8/latin1}
\lstinputlisting[language=java]{code/XMLParser.jj}

\newpage
\section{JSONParser.jj}
\label{sec:JSONParser}
\lstset{inputencoding=utf8/latin1}
\lstinputlisting[language=java]{code/JSONParser.jj}

% Volvemos a la geometría por defecto del documento
\restoregeometry