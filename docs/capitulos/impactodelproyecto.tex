\noindent Este Trabajo de Fin de Grado, a pesar de centrarse en la evaluación técnica de JavaCC como herramienta para el procesamiento de información en la asignatura PIAT, no se limita a un impacto puramente académico. Si bien no tiene una repercusión directa en áreas como la salud, la seguridad ambiental o la economía, sí presenta implicaciones relevantes en el ámbito educativo y tecnológico, con el potencial de contribuir indirectamente a un futuro más sostenible.

En el ámbito educativo, la introducción de JavaCC como herramienta alternativa para las prácticas de PIAT puede enriquecer significativamente el proceso de aprendizaje de los estudiantes. Al proporcionar una perspectiva diferente sobre el análisis y procesamiento de información, se fomenta el pensamiento crítico, la capacidad de resolución de problemas y la adaptabilidad a nuevas tecnologías, habilidades esenciales para los futuros profesionales del sector. Además, la utilización de JavaCC, una herramienta de código abierto, contribuye a la difusión y el uso de software libre en el ámbito educativo. Esto permite a los estudiantes acceder a tecnologías de vanguardia sin restricciones económicas y participar en comunidades de desarrollo colaborativo, fomentando una cultura de innovación abierta e inclusiva. Asimismo, la inclusión de JavaCC en el currículo de PIAT puede contribuir a la actualización de la asignatura, incorporando herramientas y tecnologías relevantes en el panorama actual del desarrollo de software, lo que garantiza la pertinencia y actualidad de la formación recibida.

Desde una perspectiva tecnológica, la capacidad de JavaCC para generar analizadores léxicos y sintácticos a partir de gramáticas definidas por el usuario se traduce en una mayor eficiencia en el desarrollo de software. La automatización de tareas repetitivas y la posibilidad de reutilizar código agilizan el proceso de desarrollo y permiten a los programadores centrarse en aspectos más complejos y desafiantes. La versatilidad de JavaCC, aplicable en una amplia gama de escenarios, desde el análisis de lenguajes de programación hasta el procesamiento de datos en diferentes formatos, la convierte en una herramienta valiosa para el desarrollo de soluciones tecnológicas diversas y adaptables a las necesidades cambiantes del entorno digital.

En línea con los Objetivos de Desarrollo Sostenible (ODS), este proyecto, aunque de forma indirecta, puede contribuir a su cumplimiento. Al mejorar el proceso de aprendizaje y actualizar el currículo de PIAT, se alinea con el ODS 4, ``Educación de calidad'', promoviendo oportunidades de aprendizaje a lo largo de la vida y garantizando una educación inclusiva, equitativa y de calidad. Asimismo, al fomentar el uso de herramientas eficientes y flexibles como JavaCC en el desarrollo de software, se impulsa la innovación tecnológica y la creación de infraestructuras digitales resilientes, contribuyendo al ODS 9, "Industria, innovación e infraestructura".

En definitiva, este Trabajo de Fin de Grado, más allá de su alcance técnico, plantea una reflexión sobre la importancia de la innovación educativa y la adopción de tecnologías abiertas y eficientes para el desarrollo de software. Su impacto, aunque sutil en algunos aspectos, puede tener repercusiones positivas en la formación de profesionales mejor preparados y en la construcción de un futuro más sostenible.