\section{Contexto y motivación}

\noindent El presente \pfg surge del interés personal por el procesamiento de información textual y la búsqueda de herramientas versátiles y eficientes para su análisis. Desde una temprana edad, me ha fascinado la capacidad del lenguaje para construir mundos, comunicar ideas y conectar con otras personas. En este sentido, la telemática me ha brindado la oportunidad de explorar el lenguaje desde una perspectiva más técnica y creativa, permitiéndome comprender cómo las máquinas pueden procesar y generar información textual.

A lo largo de mi formación académica, he tenido la oportunidad de aprender sobre diferentes herramientas para el procesamiento de información, como las clases de Java para formatos JSON o XML. Sin embargo, estas herramientas se limitan a un formato específico, lo que implica la necesidad de aprender un nuevo lenguaje para cada caso. Esta limitación me motivó a buscar una alternativa más universal que me permitiera abordar cualquier tipo de formato textual sin necesidad de un aprendizaje adicional.

En este contexto, descubrí JavaCC, un generador de analizadores léxicos y sintácticos que me fascinó por su potencia y flexibilidad\cite{javaccgithub}. Esta herramienta me abrió un mundo de posibilidades al permitirme concebir una forma de procesar información de forma genérica, independientemente del formato en el que se presente.

Este proyecto se presenta como una oportunidad excepcional para profundizar en mi conocimiento del lenguaje y explorar sus aplicaciones en el ámbito del procesamiento de información. Además, me motiva la idea de contribuir a la asignatura de PIAT ---perteneciente al Grado en Ingeniería Telemática de la Escuela Técnica Superior de Sistemas de Telecomunicación--- con una documentación completa y accesible que facilite el aprendizaje de esta herramienta a otros estudiantes y profesorado.

Más allá de las ventajas técnicas, este proyecto me permite conectar con mi pasión por el lenguaje y explorar su potencial en el mundo digital. Estoy convencido de que este proyecto le aportará al lector conocimientos técnicos valiosos, además de habilitarle a desarrollar sus capacidades de aprendizaje autónomo, investigación y creatividad. 

En definitiva, este proyecto representa una oportunidad única para combinar mi pasión por el lenguaje con mi formación en telemática, permitiéndome contribuir al desarrollo de nuevas herramientas y conocimientos en el ámbito del procesamiento de información textual.

\section{Objetivos}

\noindent Como se acaba de explicar, el objetivo principal del proyecto es evaluar la viabilidad a la hora de utilizar una herramienta de generadores genéricos como JavaCC para abordar las prácticas de PIAT. 

Esto implica aprender a utilizar la herramienta JavaCC de manera efectiva. En otras palabras, ahondaremos en las bases fundamentales de los analizadores y comprenderemos las virtudes y los defectos de este tipo de herramientas. Acto seguido, aplicaremos los conocimientos adquiridos al estudiar la herramienta JavaCC en prácticas específicas de la asignatura PIAT. Observaremos las carencias y limitaciones que lamentablemente se enseñan en dichos ejercicios para ofrecer una solución elegante, sencilla y extremadamente potente empleando la herramienta JavaCC.

% (Quitar los bucles y los loops de PIAT) 

Por último, el objetivo final de dicho proyecto es el de generar una documentación que sirva como guía de uso para estudiantes y profesores interesados en utilizar JavaCC en proyectos relacionados con el procesamiento de información. 

En definitiva, se busca proporcionar una solución versátil y eficaz para el análisis de gramáticas y el procesamiento de información en el ámbito de prácticas de PIAT.

\section{Restricciones}

\noindent A la hora de desarrollar y realizar del proyecto, debemos tener en cuenta las siguientes restricciones:

\begin{enumerate}
    \item El proyecto se centrará en la utilización de JavaCC como herramienta principal. 
    \item Debe garantizarse que la documentación generada de dicha herramienta sea comprensible y útil para aquellos que deseen aplicar JavaCC en proyectos similares.
    \item Las prácticas y ejercicios de la asignatura PIAT deben servir como un contexto relevante para la aplicación de JavaCC.
\end{enumerate}

Estas restricciones sirven como nuestro marco de referencia, proporcionando directrices claras sobre cómo se llevará a cabo el proyecto y asegurando que nos mantenemos fieles a nuestra visión original. Son los pilares que sostienen nuestra estrategia de proyecto y nos ayudan a mantener un enfoque claro y coherente.

A medida que avanzamos en el proyecto, estos principios y objetivos nos servirán de guía, asegurando que permanecemos en el camino correcto y que cada paso que damos nos acerca a nuestra meta final. Cada decisión que tomamos, cada desafío que enfrentamos, se evalúa en función de estas restricciones. De esta manera, nos aseguramos de que nuestro proyecto no solo cumple con los requisitos establecidos, sino que también aporta valor y conocimiento al campo de los analizadores de lenguaje.

\section{Estructura del documento}

\noindent Esta sección proporciona un esquema detallado de la estructura adoptada para la ejecución de este proyecto.

Comenzamos con el \hyperref[sec:cap1]{\textit{Capítulo 1. Introducción.}} Este capítulo sirve como punto de partida para el proyecto, estableciendo el escenario para lo que está por venir. Aquí se articula la razón detrás de la concepción del proyecto, que se deriva tanto del interés personal como de la relevancia que este proyecto tendrá en el campo. También se delinean los objetivos tangibles, en términos de los resultados que esperamos producir, como intangibles, en términos del conocimiento y la experiencia que esperamos adquirir en el proceso. Además, se establecen las bases y principios que guiarán este proyecto.

Una vez concluida esta sección, nos adentramos en el \hyperref[sec:cap2]{\textit{Capítulo 2. Estado del arte}}, en el cual exploraremos las tecnologías empleadas en la actualidad en el ámbito de los analizadores de lenguaje. Además, nos familiarizaremos con una variedad de términos técnicos y jerga sintáctica que se utilizan en este campo. Esto proporcionará al lector una visión completa de las herramientas y técnicas actuales, permitiendo una comprensión más profunda de cómo funcionan los analizadores de lenguaje y cómo se pueden aplicar en diferentes contextos. También se discutirán las ventajas y desventajas de estas tecnologías, y se presentarán tanto ejemplos prácticos como las tendencias emergentes.

El \hyperref[sec:cap3]{\textit{Capítulo 3. Desarrollo}} se dedica a la exploración en profundidad de los elementos relacionados con la implementación del proyecto. En él se detallarán el proceso de implementación, los procedimientos seguidos, el entorno de desarrollo y las pruebas, y la realización, estudio y desarrollo de las prácticas de PIAT utilizando JavaCC.

Una vez realizados los desarrollos e implementaciones correspondientes, en el \hyperref[sec:cap4]{\textit{Capítulo 4. Resultados}}, se presentarán y analizarán los resultados obtenidos a lo largo del desarrollo del proyecto. Se revelarán los resultados en términos de los objetivos establecidos al inicio del proyecto, se escudriñarán los resultados en términos de la eficacia de las técnicas y metodologías utilizadas, y se discutirá los resultados en términos de la calidad del producto final revelando las lecciones aprendidas durante el desarrollo del proyecto.

Todos estos apartados convergerán en el \hyperref[sec:cap5]{\textit{Capítulo 5, Conclusiones}}, en el cual se ofrecerá una revisión general del proyecto, presentando las conclusiones extraídas, reflexionando sobre el proceso de desarrollo y discutiendo las posibilidades de trabajo futuro.

Finalmente, el \hyperref[sec:apendice]{\textit{Apéndice}} proporciona información adicional, incluyendo detalles técnicos, documentación detallada, código fuente y otros datos relevantes. Estos detalles pueden ser útiles para aquellos lectores que deseen replicar el entorno de desarrollo o entender mejor cómo se configuró y utilizó cada herramienta. Además, se incluye el código fuente de partes significativas del desarrollo, así como cualquier información  relevante pero que no se ajusta a los capítulos principales.

\textbf{FALTA POR COMPLETAR CON TODOS LOS CAPITULOS}

% Comenzamos con el Capítulo 1. Introducción. Este capítulo sirve como punto de partida para el proyecto, estableciendo el escenario para lo que está por venir. Aquí, he articulado la razón y la inspiración detrás de la concepción de este proyecto. Esta motivación se deriva tanto de mi interés personal como de la relevancia que creo que este proyecto tendrá en el campo.

% Además, he delineado los objetivos que aspiramos a lograr a través de este proyecto. Estos objetivos son tanto tangibles, en términos de los resultados que esperamos producir, como intangibles, en términos del conocimiento y la experiencia que esperamos adquirir en el proceso.

% Finalmente, he establecido las bases y principios que guiarán este proyecto. Estos sirven como nuestro marco de referencia, proporcionando directrices claras sobre cómo se llevará a cabo el proyecto y asegurando que nos mantenemos fieles a nuestra visión original.

% A medida que avanzamos en el proyecto, estos principios y objetivos nos servirán de guía, asegurando que permanecemos en el camino correcto y que cada paso que damos nos acerca a nuestra meta final.

% Una vez concluida esta sección, nos adentramos en el Capitulo 2. Estado del arte, en el cual exploraremos las tecnologías empleadas en la actualidad en el ámbito de los analizadores de lenguaje. Además, nos familiarizaremos con una variedad de términos técnicos y jerga sintáctica que se utilizan en este campo. 

% Este capítulo proporcionará una visión completa de las herramientas y técnicas actuales, permitiendo una comprensión más profunda de cómo funcionan los analizadores de lenguaje y cómo se pueden aplicar en diferentes contextos. También se discutirán las ventajas y desventajas de estas tecnologías, proporcionando una visión equilibrada que puede informar decisiones futuras en el desarrollo de proyectos.

% Además, se presentarán ejemplos prácticos y casos de estudio para ilustrar cómo se aplican estos conceptos en el mundo real. Esto no solo ayudará a entender mejor estos términos técnicos, sino que también proporcionará al lector una base sólida para aplicar estos conocimientos en proyectos futuros.

% Finalmente, se revisarán las tendencias emergentes en el campo de los analizadores de lenguaje, proporcionando una visión de hacia dónde se dirige la industria y cómo estas tendencias podrían influir en los proyectos futuros.

% El Capítulo 3, titulado “Desarrollo”, se dedica a la exploración en profundidad de los elementos relacionados con la implementación del proyecto. En este capítulo, se desglosarán y examinarán los procedimientos adoptados, así como el entorno seleccionado para el desarrollo y las pruebas.

% En primer lugar, se detallará el proceso de implementación del proyecto. Esto incluirá una descripción de las etapas de desarrollo, desde la concepción inicial hasta la implementación final. Se discutirán las decisiones tomadas en cada etapa, proporcionando una visión clara de cómo y por qué se tomó cada decisión.

% A continuación, se describirán los procedimientos seguidos durante el desarrollo del proyecto. Esto incluirá una explicación de las metodologías y técnicas utilizadas, así como una justificación de por qué se eligieron. También se discutirán los desafíos encontrados durante el desarrollo y cómo se superaron.

% El entorno utilizado para el desarrollo y las pruebas también se examinará en detalle. Esto incluirá una descripción de las herramientas y tecnologías utilizadas, así como una justificación de por qué se eligieron. También se discutirá cómo se configuró y utilizó el entorno para facilitar el desarrollo y las pruebas.

% Además, se realizará un estudio de las prácticas de PIAT. Se examinarán estas prácticas en detalle, discutiendo cómo se aplican en el contexto del proyecto y cómo contribuyen a su éxito.

% Finalmente, se llevará a cabo el desarrollo de las prácticas de PIAT utilizando JavaCC. Se proporcionará una descripción detallada de cómo se utilizó JavaCC para implementar estas prácticas, incluyendo ejemplos de código y explicaciones de las decisiones de diseño tomadas.

% El Capítulo 4, titulado “Resultados”, se dedica a la presentación y análisis de los resultados obtenidos a lo largo del desarrollo del proyecto.

% En primer lugar, se presentarán los resultados en términos de los objetivos establecidos al inicio del proyecto. Esto incluirá una discusión sobre si se lograron estos objetivos y, si no, qué obstáculos se encontraron. Se proporcionará un análisis detallado de cada objetivo, discutiendo cómo se abordó y qué resultados se obtuvieron.

% A continuación, se analizarán los resultados en términos de la eficacia de las técnicas y metodologías utilizadas. Esto incluirá una evaluación de cómo estas técnicas contribuyeron al éxito del proyecto y si se podrían haber utilizado otras técnicas para obtener mejores resultados.

% También se discutirán los resultados en términos de la calidad del producto final. Esto incluirá una evaluación de la funcionalidad, la usabilidad y la eficiencia del producto, así como cualquier feedback recibido de los usuarios o de los evaluadores.

% Además, se presentará un análisis de los resultados obtenidos en las pruebas. Esto incluirá una discusión sobre cómo se realizaron las pruebas, qué resultados se obtuvieron y qué implicaciones tienen estos resultados para el proyecto.

% Finalmente, se discutirán las lecciones aprendidas durante el desarrollo del proyecto. Esto incluirá una reflexión sobre lo que funcionó bien, lo que podría haberse hecho mejor y qué se haría de manera diferente en futuros proyectos similares.

% Por último, el Capítulo 5, titulado “Conclusiones”, se dedica a la reflexión final y a la síntesis de los hallazgos y experiencias adquiridas a lo largo del desarrollo del proyecto.

% En primer lugar, se realizará una revisión general del proyecto, recapitulando los objetivos, la metodología, los resultados y las lecciones aprendidas. Esta revisión proporcionará una visión completa del proyecto y resaltará los aspectos más importantes.

% A continuación, se presentarán las conclusiones extraídas de los resultados del proyecto. Estas conclusiones se basarán en el análisis de los resultados y proporcionarán una visión clara de los logros del proyecto. También se discutirán las implicaciones de estos resultados para el campo de los analizadores de lenguaje y para futuros proyectos similares.

% Además, se reflexionará sobre el proceso de desarrollo del proyecto. Se discutirán los desafíos encontrados, las soluciones implementadas y las lecciones aprendidas. Esta reflexión proporcionará una visión valiosa para futuros proyectos y ayudará a mejorar las prácticas de desarrollo de proyectos.

% Finalmente, se discutirán las posibilidades de trabajo futuro basadas en este proyecto. Esto puede incluir la mejora del producto final, la expansión del proyecto a nuevas áreas o la aplicación de las técnicas y metodologías utilizadas en otros contextos.

% El Apéndice se dedica a proporcionar información adicional y complementaria que no se incluye en el cuerpo principal del proyecto, pero que puede ser útil para una comprensión más profunda o para la replicación del proyecto.

% En primer lugar, el Apéndice puede incluir detalles técnicos sobre las herramientas y tecnologías utilizadas en el proyecto. Esto puede incluir versiones de software, configuraciones específicas, scripts de instalación, entre otros. Estos detalles pueden ser útiles para aquellos que deseen replicar el entorno de desarrollo o entender mejor cómo se configuró y utilizó cada herramienta.

% A continuación, el Apéndice puede proporcionar una documentación más detallada de ciertos aspectos del proyecto. Esto puede incluir diagramas de flujo, diagramas de clases, diagramas de secuencia, entre otros. Estos diagramas pueden ayudar a visualizar la estructura y el flujo del proyecto, lo que puede ser útil para entender cómo se interconectan las diferentes partes del proyecto.

% Además, el Apéndice puede incluir el código fuente completo o partes significativas del mismo. Esto puede ser útil para aquellos que deseen estudiar el código en detalle, modificarlo o utilizarlo como base para sus propios proyectos.

% Finalmente, el Apéndice puede incluir cualquier otra información que se considere relevante pero que no se ajuste a los capítulos principales. Esto puede incluir resultados de pruebas adicionales, datos brutos, entrevistas, entre otros.