\noindent Dada la naturaleza del presente Trabajo de Fin de Grado, centrado en la evaluación de una herramienta de software libre y gratuito como JavaCC, y desarrollado utilizando recursos propios, \textbf{no ha sido necesaria una inversión económica significativa}.

% El desarrollo del proyecto se ha basado en la utilización de un ordenador personal, software libre y gratuito, y una conexión a Internet, recursos que ya se encontraban disponibles y que no generan un coste adicional para la realización del proyecto.

% Por lo tanto, se puede afirmar que el coste económico directo del proyecto ha sido de 0 €.

% Es importante destacar que, si bien no ha habido un coste económico directo, el proyecto ha requerido una inversión considerable en términos de tiempo y esfuerzo por parte del autor, dedicados a la investigación, el desarrollo, la implementación y la documentación del proyecto.

% El presente Trabajo de Fin de Grado, centrado en la evaluación de JavaCC como herramienta para el procesamiento de información en el contexto de la asignatura PIAT, se ha caracterizado por una optimización de recursos, prescindiendo de inversiones económicas significativas.

El desarrollo del proyecto se ha basado en la utilización de recursos propios, evitando así la necesidad de adquirir hardware o software específico. A continuación, se detallan los elementos clave que han permitido llevar a cabo el proyecto sin incurrir en costes directos:
\begin{itemize}
    \item Hardware: Se ha utilizado un ordenador personal, propiedad del autor, para todas las etapas del proyecto, incluyendo la investigación, el desarrollo de software, la ejecución de pruebas y la redacción de la documentación. Este enfoque ha eliminado la necesidad de adquirir o alquilar equipos informáticos adicionales.
    \item Software: Se ha optado por un ecosistema de software libre y gratuito, compuesto por:
    \begin{itemize}
        \item Sistema operativo Ubuntu 20.04 LTS: Un sistema operativo robusto y versátil que ofrece un entorno de desarrollo completo.
        \item Entorno de desarrollo Eclipse IDE: Una plataforma ampliamente utilizada para el desarrollo de aplicaciones Java, que proporciona herramientas de edición, compilación, depuración y gestión de proyectos.
        \item Herramienta JavaCC: Un generador de analizadores léxicos y sintácticos de código abierto, fundamental para el desarrollo del proyecto.
        \item Conexión a Internet: Se ha utilizado una conexión a Internet doméstica, sin coste adicional, para acceder a recursos online como documentación, tutoriales y herramientas de validación.
    \end{itemize}
\end{itemize}

En resumen, la combinación de recursos propios, software libre y gratuito, y una conexión a Internet estándar ha permitido completar el proyecto con un \textbf{coste económico directo de 0 €}.

Es fundamental destacar que, si bien el coste económico directo ha sido nulo, el proyecto ha requerido una inversión considerable en términos de tiempo y esfuerzo por parte del autor. Esta inversión se ha materializado en la dedicación a la investigación, el aprendizaje de nuevas tecnologías, el desarrollo del software, la realización de pruebas exhaustivas y la elaboración de una documentación completa y detallada.