Las cuatro prácticas realizadas han permitido evaluar el potencial de JavaCC como herramienta única para el procesamiento de información en el contexto de la asignatura PIAT. A continuación, se presentan los resultados obtenidos en cada una de ellas, destacando las ventajas y particularidades de la utilización de JavaCC.

\phantom{text}

\noindent \textbf{Práctica 2: Análisis de ficheros de log}

\phantom{text}


La segunda práctica se centró en el análisis de archivos de log de un sistema de correo electrónico. El objetivo principal era la extracción de información estadística utilizando expresiones regulares. En este contexto, se implementó un analizador léxico y sintáctico con JavaCC que demostró ser significativamente más eficiente que el uso de expresiones regulares tradicionales.

Los resultados de las pruebas de rendimiento evidenciaron una reducción del tiempo de ejecución del 450\% al utilizar JavaCC. Esta mejora se atribuye a la eficiencia en la compilación de gramáticas, la menor sobrecarga de procesamiento y la optimización del código generado por JavaCC. La capacidad de definir gramáticas estructuradas y la ejecución de acciones específicas según el tipo de traza identificado, convirtieron a JavaCC en una herramienta mucho más eficiente para el análisis de grandes volúmenes de datos.

\phantom{text}

\noindent \textbf{Práctica 3: Análisis de archivos XML}

\phantom{text}

En la tercera práctica, se abordó el procesamiento de archivos XML. Se implementó un analizador basado en JavaCC para extraer información específica de un conjunto de datos del Ayuntamiento de Madrid. La principal ventaja observada en esta práctica fue la simplicidad en la generación del analizador y su capacidad de manejar recursividad en los documentos XML.

A diferencia de SAX, que requeriría una modificación sustancial del código ante un cambio en la profundidad de la recursividad del archivo XML, JavaCC permite definir la gramática de manera que se adapte a diferentes niveles de anidamiento sin necesidad de reescribir el código principal. Esta característica aporta flexibilidad y facilita el mantenimiento del código.

\phantom{text}

\noindent \textbf{Práctica 4: Procesamiento de JSON con GSON}

\phantom{text}

[Aquí incluir la información sobre la práctica 4, describiendo los resultados obtenidos y cómo JavaCC se compara con GSON en este caso particular. ]


\phantom{text}

\noindent \textbf{Práctica 5: Consultas con XPath}

\phantom{text}

[Aquí incluir la información sobre la práctica 5, describiendo los resultados obtenidos y cómo JavaCC se compara con XPath en este caso particular. ]

En resumen, las cuatro prácticas han permitido demostrar la versatilidad y eficiencia de JavaCC como herramienta única para el procesamiento de información en PIAT. Si bien existen herramientas especializadas que pueden resultar más eficientes en casos específicos, la versatilidad, eficiencia y capacidad de JavaCC para manejar estructuras complejas lo convierten en una alternativa a considerar, especialmente cuando se requiere un alto grado de control sobre el proceso de análisis o se necesita una solución adaptable a diferentes formatos y estructuras de datos. La capacidad de definir gramáticas, la eficiencia en la compilación y la optimización del código generado, convierten a JavaCC en una alternativa sólida y eficiente a las herramientas tradicionales utilizadas en la asignatura.